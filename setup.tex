\usepackage{ifxetex}

\usepackage{geometry}
\usepackage{extsizes}

\usepackage{amsthm,amsfonts,amssymb,amscd}
\usepackage[intlimits, sumlimits]{amsmath}
\usepackage[nodisplayskipstretch]{setspace}

\usepackage{mathtools} 		% for \underbracket
\usepackage{calc}			% for \widthof

%\usepackage{natbib}



\ifxetex
	\usepackage{unicode-math}
	\usepackage{polyglossia}
	\usepackage{fontspec}
\else
	\usepackage{cmap}                                                     
	\usepackage[T2A]{fontenc}                        
	\usepackage[utf8x]{inputenc}
	\usepackage[english, russian]{babel}
\fi

\usepackage[mode=text]{siunitx}

\usepackage[tableposition=top]{caption}
\usepackage{subcaption}
\usepackage{indentfirst}

\usepackage[version=4]{mhchem}						% Typesetting chemestry
\usepackage{titlepage}			% formate custom titlepage
\usepackage[figuresonly,nomarkers,figlist]{endfloat} % figures at the end of the file

%%%Settings

\geometry{a4paper,top=2.5cm,bottom=2.5cm,left=3cm,right=2cm}
\onehalfspacing

\ifxetex
	\setmainlanguage{russian}
	\setotherlanguage{english}
	\defaultfontfeatures{Ligatures=TeX,Mapping=tex-text}
	\setmainfont{Times New Roman}
  % fonts for figures
	\setmathfont[version={sfmath}]{Arial Italic}
	\setmathfont[version={normalmath}]{Latin Modern Math}
	\setmathfont{Latin Modern Math}
	\setsansfont{Arial}
\else
	% patch \efloat@iwrite to use \protected@write
	\makeatletter
	\renewcommand\efloat@iwrite[1]{%
		\immediate\expandafter\protected@write\csname efloat@post#1\endcsname{}}
	\makeatother
\fi

\sisetup{
	exponent-product = \cdot,
	output-decimal-marker = {.},
	separate-uncertainty = true,
	list-separator = {; },
	list-final-separator = { и },
	list-pair-separator = { и },
	list-units=single,
	valuesep = {~},
	range-phrase = \,--\,,
	range-units=single,
}

\usepackage{titlesec} 
\titleformat{\section}
{\normalfont\normalsize\bfseries}{\thesection.~}{1em}{}

\titleformat{\subsection}
{\normalfont\normalsize\bfseries}{\thesubsection}{1em}{}

\titleformat{\subsubsection}
{\normalfont\normalsize\bfseries}{\thesubsubsection}{1em}{}

\captionsetup[figure]{labelsep=endash,justification=raggedleft,singlelinecheck=off, skip=50pt}
\captionsetup[subfigure]{labelfont={normalsize,rm,it}, margin=10pt, justification=raggedleft, skip=-18pt}

\makeatletter
\def\thesubfigure{\textit{\alph{subfigure}}}
\providecommand\thefigsubsep{,\,}
\def\p@subfigure{\@nameuse{thefigure}\thefigsubsep}
\makeatother

\usepackage[titles]{tocloft}
\renewcommand{\cftfigpresnum}{\figurename~}		% Формат списка иллюстраций
\renewcommand{\cftfignumwidth}{4em}
\renewcommand{\cftbeforefigskip}{14pt}
\cftpagenumbersoff{figure}

\addto\captionsrussian{%
    \renewcommand\listfigurename{Список подрисуночных подписей}}
	
\doublespacing 				 %new requirements

% add brakets under abbriviations, equations and units
\newcommand\ph[1]{%
    \raisebox{-0.4em}{\rule{.5pt}{3pt}\rule{\widthof{#1}}{.5pt}\rule{.5pt}{3pt}}%
    \llap{#1}}


\let\oldce\ce
\renewcommand{\ce}[1]{\ph{\oldce{#1}}}

\let\oldsi\si
\renewcommand{\si}[1]{\ph{\oldsi{#1}}}

\renewcommand{\SI}[2]{\num{#1}\,\ph{\oldsi{#2}}}

\catcode`\$=\active
\gdef$#1${\ph{\(#1\)}}